\documentclass{article}

\usepackage{amsmath}
\usepackage[russian]{babel}

\usepackage{tikz}
\usetikzlibrary{graphs, babel, quotes, calc}

\usepackage[margin=2.5cm]{geometry}


\title{Курсовая работа по дискретной математике\\Вторая задача}
\author{Клименко В. М. -- М8О-103Б-22 -- 11 вариант}
\date{Март, 1523}

\begin{document}

\maketitle

\section*{Дано}
Граф:

\begin{center}
  \tikz {
    \node [circle, draw] (1) at (5,-3) {1};
    \node [circle, draw] (2) at (6,-1) {2};
    \node [circle, draw] (3) at (3,-2) {3};
    \node [circle, draw] (4) at (3, 0) {4};
    \node [circle, draw] (5) at (0,-1) {5};
    \node [circle, draw] (6) at (1,-3) {6};
    \graph {
      (5) -- {(3), (4)} -- (2);
      (5) -- (6) -- (1) -- (2);
    };
  }
\end{center}

\section*{Задание}
Используя алгоритм Терри, определить замкнутый маршрут, проходящий ровно по два раза
(по одному в каждом направлении) через каждое ребро графа

\section*{Решение}
\begin{center}
  \tikz {
    \node [circle, draw, label={180:{*}}] (1) at (5,-3) {1};
    \node [circle, draw, label={170:{*}}] (2) at (6,-1) {2};
    \node [circle, draw, label={0  :{*}}] (3) at (3,-2) {3};
    \node [circle, draw, label={180:{*}}] (4) at (3, 0) {4};
    \node [circle, draw, label={-5 :{*}}] (5) at (0,-1) {5};
    \node [circle, draw, label={110:{*}}] (6) at (1,-3) {6};
    \graph {
      (5) -- {(3), (4)} -- (2);
      (5) -- (6) -- (1) -- (2);
    };
    \draw[-to] ($(5)!10mm! 15:(4)$) to ($(4)!10mm!-15:(5)$);
    \draw[-to] ($(5)!10mm! 15:(4)$) to ($(4)!10mm!-15:(5)$);
    \draw[-to] ($(4)!10mm! 15:(2)$) to ($(2)!10mm!-15:(4)$);
    \draw[-to] ($(2)!10mm!-15:(3)$) to ($(3)!10mm! 15:(2)$);
    \draw[-to] ($(3)!10mm!-15:(5)$) to ($(5)!10mm! 15:(3)$);
    \draw[-to] ($(5)! 7mm! 15:(6)$) to ($(6)! 7mm!-15:(5)$);
    \draw[-to] ($(6)!10mm! 15:(1)$) to ($(1)!10mm!-15:(6)$);
    \draw[-to] ($(1)! 7mm! 15:(2)$) to ($(2)! 7mm!-15:(1)$);

    \draw[-to] ($(2)! 7mm! 15:(1)$) to ($(1)! 7mm!-15:(2)$);
    \draw[-to] ($(1)!10mm! 15:(6)$) to ($(6)!10mm!-15:(1)$);
    \draw[-to] ($(6)! 7mm! 15:(5)$) to ($(5)! 7mm!-15:(6)$);
    \draw[-to] ($(5)!10mm!-15:(3)$) to ($(3)!10mm! 15:(5)$);
    \draw[-to] ($(3)!10mm!-15:(2)$) to ($(2)!10mm! 15:(3)$);
    \draw[-to] ($(2)!10mm! 15:(4)$) to ($(4)!10mm!-15:(2)$);
    \draw[-to] ($(4)!10mm! 15:(5)$) to ($(5)!10mm!-15:(4)$);
  }
\end{center}

\section*{Ответ}
В итоге получился такой путь:
$5-4-2-3-5-6-1-2-1-6-5-3-2-4-5$


\end{document}

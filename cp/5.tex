\documentclass{article}

\usepackage{amsmath}
\usepackage[russian]{babel}
\usepackage[margin=2.5cm]{geometry}
\usepackage{booktabs}

\usepackage{tikz}
\usetikzlibrary{graphs, babel, quotes, calc}


\title{Курсовая работа по дискретной математике\\Пятая задача}
\author{Клименко В. М. -- М8О-103Б-22 -- 11 вариант}
\date{Май, 2023}

\begin{document}
\maketitle


\section*{Дано}
\begin{center}
    \begin{tikzpicture}[circ/.style={circle, draw, fill=black}]
        \path[nodes={circ}]
            (0, 0) node(1)  {}
            (0, 2) node(2)  {}
            (0, 4) node(3)  {}
            (2, 0) node(4)  {}
            (2, 2) node(5)  {}
            (2, 4) node(6)  {}
            (4, 0) node(7)  {}
            (4, 2) node(8)  {}
            (4, 4) node(9)  {}
            (6, 0) node(10) {}
            (6, 2) node(11) {}
            (6, 4) node(12) {}
        ;
        \draw[nodes={auto}]
            (1)  -- node{3} (2)  -- node{4} (3)
            (4)  -- node{5} (5)  -- node{5} (6)
            (7)  -- node{5} (8)  -- node{5} (9)
            (10) -- node{7} (11) -- node{8} (12)

            (3) -- node{2} (6) -- node{2} (9) -- node{4} (12)
            (2) -- node{1} (5) -- node{6} (8) -- node{3} (11)
            (1) -- node{5} (4) -- node{7} (7) -- node{6} (10)
        ;
    \end{tikzpicture}
\end{center}

\section*{Задание}
Найти остовное дерево с минимальной суммой длин входящих в него ребер

\section*{Решение}
\subsection*{1}
\begin{center}
    \begin{tikzpicture}[circ/.style={circle, draw, fill=black}]
        \path[nodes={circ}]
            (0, 0) node(1)  {}
            (0, 2) node(2)  {}
            (0, 4) node(3)  {}
            (2, 0) node(4)  {}
            (2, 2) node(5)  {}
            (2, 4) node(6)  {}
            (4, 0) node(7)  {}
            (4, 2) node(8)  {}
            (4, 4) node(9)  {}
            (6, 0) node(10) {}
            (6, 2) node(11) {}
            (6, 4) node(12) {}
        ;
    \end{tikzpicture}
\end{center}

\subsection*{2}
\begin{center}
    \begin{tikzpicture}[circ/.style={circle, draw, fill=black}]
        \path[nodes={circ}]
            (0, 0) node(1)  {}
            (0, 2) node(2)  {}
            (0, 4) node(3)  {}
            (2, 0) node(4)  {}
            (2, 2) node(5)  {}
            (2, 4) node(6)  {}
            (4, 0) node(7)  {}
            (4, 2) node(8)  {}
            (4, 4) node(9)  {}
            (6, 0) node(10) {}
            (6, 2) node(11) {}
            (6, 4) node(12) {}
        ;
        \draw[nodes={auto}]
            (1)  -- node{3} (2)  -- node{4} (3)

            (2) -- node{1} (5)
        ;
    \end{tikzpicture}
\end{center}

\subsection*{3}
\begin{center}
    \begin{tikzpicture}[circ/.style={circle, draw, fill=black}]
        \path[nodes={circ}]
            (0, 0) node(1)  {}
            (0, 2) node(2)  {}
            (0, 4) node(3)  {}
            (2, 0) node(4)  {}
            (2, 2) node(5)  {}
            (2, 4) node(6)  {}
            (4, 0) node(7)  {}
            (4, 2) node(8)  {}
            (4, 4) node(9)  {}
            (6, 0) node(10) {}
            (6, 2) node(11) {}
            (6, 4) node(12) {}
        ;
        \draw[nodes={auto}]
            (1)  -- node{3} (2)  -- node{4} (3)

            (3) -- node{2} (6)
            (2) -- node{1} (5) -- node{6} (8)
            (1) -- node{5} (4)
        ;
    \end{tikzpicture}
\end{center}

\subsection*{4}
\begin{center}
    \begin{tikzpicture}[circ/.style={circle, draw, fill=black}]
        \path[nodes={circ}]
            (0, 0) node(1)  {}
            (0, 2) node(2)  {}
            (0, 4) node(3)  {}
            (2, 0) node(4)  {}
            (2, 2) node(5)  {}
            (2, 4) node(6)  {}
            (4, 0) node(7)  {}
            (4, 2) node(8)  {}
            (4, 4) node(9)  {}
            (6, 0) node(10) {}
            (6, 2) node(11) {}
            (6, 4) node(12) {}
        ;
        \draw[nodes={auto}]
            (1)  -- node{3} (2)  -- node{4} (3)
            (7)  -- node{5} (8)  -- node{5} (9)

            (3) -- node{2} (6) (9) -- node{4} (12)
            (2) -- node{1} (5) -- node{6} (8) -- node{3} (11)
            (1) -- node{5} (4) (7) -- node{6} (10)
        ;
    \end{tikzpicture}
\end{center}
$L = 1 + 3 + 4 + 2 + 5 + 6 + 5 + 5 + 6 + 3 + 4 = 44$


\section*{Ответ}
\begin{center}
    \begin{tikzpicture}[circ/.style={circle, draw, fill=black}]
        \path[nodes={circ}]
            (0, 0) node(1)  {}
            (0, 2) node(2)  {}
            (0, 4) node(3)  {}
            (2, 0) node(4)  {}
            (2, 2) node(5)  {}
            (2, 4) node(6)  {}
            (4, 0) node(7)  {}
            (4, 2) node(8)  {}
            (4, 4) node(9)  {}
            (6, 0) node(10) {}
            (6, 2) node(11) {}
            (6, 4) node(12) {}
        ;
        \draw[nodes={auto}]
            (1)  -- node{3} (2)  -- node{4} (3)
            (7)  -- node{5} (8)  -- node{5} (9)

            (3) -- node{2} (6) (9) -- node{4} (12)
            (2) -- node{1} (5) -- node{6} (8) -- node{3} (11)
            (1) -- node{5} (4) (7) -- node{6} (10)
        ;
    \end{tikzpicture}
\end{center}
$L = 1 + 3 + 4 + 2 + 5 + 6 + 5 + 5 + 6 + 3 + 4 = 44$


\end{document}

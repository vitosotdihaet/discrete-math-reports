\documentclass{article}

\usepackage{amsmath}
\usepackage{ragged2e}
\usepackage[russian]{babel}
\usepackage[margin=2.5cm]{geometry}

\setcounter{MaxMatrixCols}{20}


\title{Курсовая работа по дискретной математике\\Третья задача}
\author{Клименко В. М. -- М8О-103Б-22 -- 11 вариант}
\date{Март, 2023}


\RaggedRight

\begin{document}
\maketitle

\section*{Дано}
Матрица смежности $A$ орграфа:
$$
\begin{pmatrix}
  % 1    2    3    4    5    6    7
    0 && 0 && 0 && 1 && 0 && 1 && 0 \\ % 1
    1 && 0 && 1 && 0 && 1 && 0 && 1 \\ % 2
    0 && 1 && 0 && 1 && 1 && 1 && 0 \\ % 3
    1 && 0 && 0 && 0 && 1 && 0 && 0 \\ % 4
    1 && 0 && 1 && 1 && 0 && 1 && 0 \\ % 5
    0 && 0 && 0 && 1 && 1 && 0 && 0 \\ % 6
    1 && 0 && 1 && 0 && 1 && 1 && 0 \\ % 7
\end{pmatrix}
$$


\section*{Задание}
Используя алгоритм ''фронта волны'', найти все минимальные
пути из первой вершины в последнюю орграфа, заданного матрицей
смежности $A$


\section*{Решение}
Находим фронты волны:

\begin{align*}
    w_0(x_1) &= \{x_1\} \\
    w_1(x_1) &= \Gamma_{x_1} = \{x_4, x_6\} \\
    w_2(x_1) &= \Gamma_{x_4, x_6} = \{x_1, x_5, x_4\} \\
    w_3(x_1) &= \Gamma_{x_5} = \{x_1, x_3, x_4, x_6\} \\
    w_4(x_1) &= \Gamma_{x_3} = \{x_2, x_4, x_5, x_6\} \\
    w_5(x_1) &= \Gamma_{x_2} = \{x_1, x_3, x_5, x_7\}
\end{align*}

Нашли $x_7$ на шестом шаге, следовательно путь состоит из шести вершин из $x_1$ в $x_7$ равен:
$$
x_1 \rightarrow z_1 \rightarrow z_2 \rightarrow z_3 \rightarrow z_4 \rightarrow x_7,
$$
где

\begin{align*}
    \text{1.   } z_4 \in w_4(x_1) \cap \Gamma^{-1}_{x_7} &=
\{x_2, x_4, x_5, x_6\} \cap \{x_2\} = \{x_2\} \Rightarrow z_5 = x_2 \\
    \text{2.   } z_3 \in w_3(x_1) \cap \Gamma^{-1}_{x_2} &=
\{x_1, x_3, x_4, x_6\} \cap \{x_3\} = \{x_3\} \Rightarrow z_4 = x_3 \\
    \text{3.   } z_2 \in w_2(x_1) \cap \Gamma^{-1}_{x_3} &=
\{x_1, x_5, x_4\} \cap \{x_2, x_5, x_7\} = \{x_5\} \Rightarrow z_3 = x_5 \\
    \text{4.   } z_1 \in w_1(x_1) \cap \Gamma^{-1}_{x_5} &=
\{x_4, x_6\} \cap \{x_2, x_3, x_4, x_6\} = \{x_4, x_6\} \Rightarrow
z_1 \in \{x_4, x_6\} \\
    \text{5.1. } x_1 \in w_0(x_1) \cap \Gamma^{-1}_{x_4} &=
\{x_1\} \cap \{x_1, x_3, x_5, x_6\} = \{x_1\} \\
    \text{5.2. } x_1 \in w_0(x_1) \cap \Gamma^{-1}_{x_6} &=
\{x_1\} \cap \{x_1, x_3, x_5, x_7\} = \{x_1\} \\
\end{align*}

Из этого следует, что всего найдено два минимальных пути из $x_1$ в $x_7$:
\begin{align*}
    x_1 \rightarrow x_4 \rightarrow x_5 \rightarrow x_3 \rightarrow x_2 \rightarrow x_7 \\
    x_1 \rightarrow x_6 \rightarrow x_5 \rightarrow x_3 \rightarrow x_2 \rightarrow x_7
\end{align*}


\section*{Ответ}
Найдено два кратчайших пути:
\begin{align*}
    x_1 \rightarrow x_4 \rightarrow x_5 \rightarrow x_3 \rightarrow x_2 \rightarrow x_7 \\
    x_1 \rightarrow x_6 \rightarrow x_5 \rightarrow x_3 \rightarrow x_2 \rightarrow x_7
\end{align*}



\end{document}

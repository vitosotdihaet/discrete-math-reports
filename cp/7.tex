\documentclass{article}

\usepackage{amsmath}
\usepackage[russian]{babel}
\usepackage[margin=2.5cm]{geometry}
\usepackage{booktabs}

\usepackage{tikz}
\usetikzlibrary{graphs, babel, quotes, calc, arrows.meta}

\title{Курсовая работа по дискретной математике\\Седьмая задача}
\author{Клименко В. М. -- М8О-103Б-22 -- 11 вариант}
\date{Май, 2023}

\begin{document}
\maketitle

\section*{Дано}
\begin{center}
    \begin{tikzpicture}[->, circ/.style={circle, draw}]
        \path[nodes={circ}]
            (0, 3) node(1) {$v_1$}
            (2, 5) node(2) {$v_2$}
            (2, 1) node(3) {$v_3$}
            (4, 6) node(4) {$v_4$}
            (4, 3) node(5) {$v_5$}
            (4, 0) node(6) {$v_6$}
            (6, 5) node(7) {$v_7$}
            (6, 1) node(8) {$v_8$}
            (8, 3) node(9) {$v_9$}
        ;
        \draw[->, nodes={circle, fill=white}, thick] (1) -- node{4} (2);
        \draw[->, nodes={circle, fill=white}, thick] (1) -- node{3} (3);
        \draw[->, nodes={circle, fill=white}, thick] (1) to[in=-120, out=20] node{9} (4);
        \draw[->, nodes={circle, fill=white}, thick] (1) -- node{5} (5);
        \draw[->, nodes={circle, fill=white}, thick] (1) to[in=120, out=-20] node{10} (6);

        \draw[->, nodes={circle, fill=white}, thick] (2) -- node{4} (4);
        \draw[->, nodes={circle, fill=white}, thick] (2) -- node{8} (5);
        \draw[->, nodes={circle, fill=white}, thick] (3) -- node{5} (5);
        \draw[->, nodes={circle, fill=white}, thick] (3) -- node{3} (6);

        \draw[->, nodes={circle, fill=white}, thick] (4) -- node{8} (7);
        \draw[->, nodes={circle, fill=white}, thick] (5) -- node{5} (7);
        \draw[->, nodes={circle, fill=white}, thick] (5) -- node{4} (9);
        \draw[->, nodes={circle, fill=white}, thick] (5) -- node{4} (8);
        \draw[->, nodes={circle, fill=white}, thick] (6) -- node{10} (8);

        \draw[->, nodes={circle, fill=white}, thick] (7) -- node{12} (9);
        \draw[->, nodes={circle, fill=white}, thick] (8) -- node{15} (9);
        ;
    \end{tikzpicture}
\end{center}


\section*{Задание}
Пострить максимальный поток по транспортной сети


\section*{Решение}
Выбираем нулевой поток в качестве начального $\varphi_{ij} = 0, \forall i, j$
\\
Составим первый ненулевой поток:
\begin{center}
    \begin{tikzpicture}[->, circ/.style={circle, draw}]
        \path[nodes={circ}]
            (0, 3) node(1) {$v_1$}
            (2, 5) node(2) {$v_2$}
            (2, 1) node(3) {$v_3$}
            (4, 6) node(4) {$v_4$}
            (4, 3) node(5) {$v_5$}
            (4, 0) node(6) {$v_6$}
            (6, 5) node(7) {$v_7$}
            (6, 1) node(8) {$v_8$}
            (8, 3) node(9) {$v_9$}
        ;
        \draw[->, nodes={circle, fill=white}, thick] (1) -- node{4} (2);
        \draw[->, nodes={circle, fill=white}, thick] (1) -- node{0} (3);
        \draw[->, nodes={circle, fill=white}, thick] (1) to[in=-120, out=20] node{0} (4);
        \draw[->, nodes={circle, fill=white}, thick] (1) -- node{0} (5);
        \draw[->, nodes={circle, fill=white}, thick] (1) to[in=120, out=-20] node{0} (6);

        \draw[->, nodes={circle, fill=white}, thick] (2) -- node{4} (4);
        \draw[->, nodes={circle, fill=white}, thick] (2) -- node{0} (5);
        \draw[->, nodes={circle, fill=white}, thick] (3) -- node{0} (5);
        \draw[->, nodes={circle, fill=white}, thick] (3) -- node{0} (6);

        \draw[->, nodes={circle, fill=white}, thick] (4) -- node{4} (7);
        \draw[->, nodes={circle, fill=white}, thick] (5) -- node{0} (7);
        \draw[->, nodes={circle, fill=white}, thick] (5) -- node{0} (9);
        \draw[->, nodes={circle, fill=white}, thick] (5) -- node{0} (8);
        \draw[->, nodes={circle, fill=white}, thick] (6) -- node{0} (8);

        \draw[->, nodes={circle, fill=white}, thick] (7) -- node{4} (9);
        \draw[->, nodes={circle, fill=white}, thick] (8) -- node{0} (9);
        ;
    \end{tikzpicture}
\end{center}

Существуют пути, не содержащие насыщенных дуг, следовательно продолжаем увеличивать поток
на этих путях:
\begin{center}
    \begin{tikzpicture}[->, circ/.style={circle, draw}]
        \path[nodes={circ}]
            (0, 3) node(1) {$v_1$}
            (2, 5) node(2) {$v_2$}
            (2, 1) node(3) {$v_3$}
            (4, 6) node(4) {$v_4$}
            (4, 3) node(5) {$v_5$}
            (4, 0) node(6) {$v_6$}
            (6, 5) node(7) {$v_7$}
            (6, 1) node(8) {$v_8$}
            (8, 3) node(9) {$v_9$}
        ;
        \draw[->, nodes={circle, fill=white}, thick] (1) -- node{4} (2);
        \draw[->, nodes={circle, fill=white}, thick] (1) -- node{3} (3);
        \draw[->, nodes={circle, fill=white}, thick] (1) to[in=-120, out=20] node{4} (4);
        \draw[->, nodes={circle, fill=white}, thick] (1) -- node{5} (5);
        \draw[->, nodes={circle, fill=white}, thick] (1) to[in=120, out=-20] node{7} (6);

        \draw[->, nodes={circle, fill=white}, thick] (2) -- node{4} (4);
        \draw[->, nodes={circle, fill=white}, thick] (2) -- node{0} (5);
        \draw[->, nodes={circle, fill=white}, thick] (3) -- node{0} (5);
        \draw[->, nodes={circle, fill=white}, thick] (3) -- node{3} (6);

        \draw[->, nodes={circle, fill=white}, thick] (4) -- node{8} (7);
        \draw[->, nodes={circle, fill=white}, thick] (5) -- node{0} (7);
        \draw[->, nodes={circle, fill=white}, thick] (5) -- node{4} (9);
        \draw[->, nodes={circle, fill=white}, thick] (5) -- node{1} (8);
        \draw[->, nodes={circle, fill=white}, thick] (6) -- node{10} (8);

        \draw[->, nodes={circle, fill=white}, thick] (7) -- node{8} (9);
        \draw[->, nodes={circle, fill=white}, thick] (8) -- node{11} (9);
        ;
    \end{tikzpicture}
\end{center}

В потоке нет путей без насыщенных дуг, следовательно ищем увеличивающие цепи:
\begin{enumerate}
    \item $v_1 - v_4 - v_2 - v_5 - v_7 - v_9$: $\Delta_1 = \min\{9 - 4, 4, 8, 5, 12 - 8\} = 4$
    \item $v_1 - v_6 - v_3 - v_5 - v_8 - v_9$: $\Delta_2 = \min\{10 - 3, 3, 5, 4, 15 - 10\} = 3$
\end{enumerate}
Строим максимальный поток:
\begin{center}
    \begin{tikzpicture}[->, circ/.style={circle, draw}]
        \path[nodes={circ}]
            (0, 3) node(1) {$v_1$}
            (2, 5) node(2) {$v_2$}
            (2, 1) node(3) {$v_3$}
            (4, 6) node(4) {$v_4$}
            (4, 3) node(5) {$v_5$}
            (4, 0) node(6) {$v_6$}
            (6, 5) node(7) {$v_7$}
            (6, 1) node(8) {$v_8$}
            (8, 3) node(9) {$v_9$}
        ;
        \draw[->, nodes={circle, fill=white}, thick] (1) -- node{4} (2);
        \draw[->, nodes={circle, fill=white}, thick] (1) -- node{3} (3);
        \draw[->, nodes={circle, fill=white}, thick] (1) to[in=-120, out=40] node{8} (4);
        \draw[->, nodes={circle, fill=white}, thick] (1) -- node{5} (5);
        \draw[->, nodes={circle, fill=white}, thick] (1) to[in=120, out=-40] node{10} (6);

        \draw[->, nodes={circle, fill=white}, thick] (2) -- node{4-4} (4);
        \draw[->, nodes={circle, fill=white}, thick] (2) -- node{0+4} (5);
        \draw[->, nodes={circle, fill=white}, thick] (3) -- node{0+3} (5);
        \draw[->, nodes={circle, fill=white}, thick] (3) -- node{3-3} (6);

        \draw[->, nodes={circle, fill=white}, thick] (4) -- node{8-4+4} (7);
        \draw[->, nodes={circle, fill=white}, thick] (5) -- node{0+4} (7);
        \draw[->, nodes={circle, fill=white}, thick] (5) -- node{4} (9);
        \draw[->, nodes={circle, fill=white}, thick] (5) -- node{1+3} (8);
        \draw[->, nodes={circle, fill=white}, thick] (6) -- node{10-3+3} (8);

        \draw[->, nodes={circle, fill=white}, thick] (7) -- node{12} (9);
        \draw[->, nodes={circle, fill=white}, thick] (8) -- node{11+3} (9);
        ;
    \end{tikzpicture}
\end{center}
Величина максимального потока $\Phi = 12 + 4 + 14 = 30$

\section*{Ответ}
$\Phi = 30$


\end{document}

\documentclass[a4paper,12pt]{article} % добавить leqno в [] для нумерации слева
\usepackage[a4paper,top=1.3cm,bottom=2cm,left=1.5cm,right=1.5cm,marginparwidth=0.75cm]{geometry}
%%% Работа с русским языком
\usepackage{cmap}					% поиск в PDF
\usepackage{mathtext} 				% русские буквы в фомулах
\usepackage[T2A]{fontenc}			% кодировка
\usepackage[utf8]{inputenc}			% кодировка исходного текста
\usepackage[english,russian]{babel}	% локализация и переносы
\usepackage{multirow}

\usepackage{graphicx}

\usepackage{wrapfig}
\usepackage{tabularx}
\usepackage{float}
\usepackage{graphicx}

\usepackage{hyperref}
\usepackage[rgb]{xcolor}
\hypersetup{
colorlinks=true,urlcolor=blue
}

\documentclass[a4paper, 10pt]{article}%тип документа

%отступы
\usepackage[left=2cm,right=2cm,top=2cm,bottom=3cm,bindingoffset=0cm]{geometry}

%Русский язык
\usepackage[T2A]{fontenc} %кодировка
\usepackage[utf8]{inputenc} %кодировка исходного кода
\usepackage[english,russian]{babel} %локализация и переносы

%Вставка картинок
\usepackage{graphicx}
\graphicspath{{pictures/}}
\DeclareGraphicsExtensions{.pdf,.png,.jpg}

%Графики
\usepackage{pgfplots}
\pgfplotsset{compat=1.9}
\usepackage{float}

%Математика
\usepackage{amsmath, amsfonts, amssymb, amsthm, mathtools}

%%% Дополнительная работа с математикой
\usepackage{amsmath,amsfonts,amssymb,amsthm,mathtools} % AMS
\usepackage{icomma} % "Умная" запятая: $0,2$ --- число, $0, 2$ --- перечисление

%% Номера формул
\mathtoolsset{showonlyrefs=true} % Показывать номера только у тех формул, на которые есть \eqref{} в тексте.

%% Шрифты
\usepackage{euscript}	 % Шрифт Евклид
\usepackage{mathrsfs} % Красивый матшрифт

%% Свои команды
\DeclareMathOperator{\sgn}{\mathop{sgn}}

%% Перенос знаков в формулах (по Львовскому)
\newcommand*{\hm}[1]{#1\nobreak\discretionary{}
{\hbox{$\mathsurround=0pt #1$}}{}}

\title{Курсовая работа по дискретной математике\\Первая задача}
\author{Клименко В. М. -- М8О-103Б-22 -- 11 вариант}
\date{Март, 2023}

\begin{document}

\maketitle
               
\section*{Дано}
Матрица смежности орграфа
$$
A =
\begin{pmatrix}
  0 && 1 && 0 && 0 \\
  1 && 0 && 1 && 0 \\
  0 && 1 && 0 && 0 \\
  1 && 1 && 1 && 0
\end{pmatrix}
$$

\section*{Найти}
\begin{enumerate}
\item матрицу односторонней связности
\item матрицу сильной связности
\item компоненты сильной связности 
\item матрицу контуров
\end{enumerate}

\section*{Решение}
Найдем матрицу односторонней связности при помощи итерационного алгоритма:

\subsection*{1.}

\begin{enumerate}
\item
$
  T^{(0)} = E \lor A =
  \begin{pmatrix}
  1 && 1 && 0 && 0 \\
  1 && 1 && 1 && 0 \\
  0 && 1 && 1 && 0 \\
  1 && 1 && 1 && 1
  \end{pmatrix}
$
\item
$
  T^{(1)} = ||t^{(1)}_{ij}||, t^{(1)}_{ij} = t^{(0)}_{ij} \lor (t^{(0)}_{i1} \& t^{(0)}_{1j}) =
  \begin{pmatrix}
  1 && 1 && 0 && 0 \\
  1 && 1 && 1 && 0 \\
  0 && 1 && 1 && 0 \\
  1 && 1 && 1 && 1
  \end{pmatrix}
$
  \begin{enumerate}[label=]
  \item $t_{13} = 0$
  \item $t_{14} = 0$
  \item $t_{24} = 0$
  \item $t_{31} = 0$
  \item $t_{34} = 0$
  \end{enumerate}
\item
$
  T^{(2)} = ||t^{(2)}_{ij}||, t^{(2)}_{ij} = t^{(1)}_{ij} \lor (t^{(1)}_{i2} \& t^{(1)}_{2j}) =
  \begin{pmatrix}
  1 && 1 && 1 && 0 \\
  1 && 1 && 1 && 0 \\
  1 && 1 && 1 && 0 \\
  1 && 1 && 1 && 1
  \end{pmatrix}
$
  \begin{enumerate}[label=]
  \item $t_{13} = 1$
  \item $t_{14} = 0$
  \item $t_{24} = 0$
  \item $t_{31} = 1$
  \item $t_{34} = 0$
  \end{enumerate}
\item
$
  T^{(3)} = ||t^{(3)}_{ij}||, t^{(3)}_{ij} = t^{(2)}_{ij} \lor (t^{(2)}_{i3} \& t^{(2)}_{3j}) =
  \begin{pmatrix}
  1 && 1 && 1 && 0 \\
  1 && 1 && 1 && 0 \\
  1 && 1 && 1 && 0 \\
  1 && 1 && 1 && 1
  \end{pmatrix}
$
  \begin{enumerate}[label=]
  \item $t_{14} = 0$
  \item $t_{24} = 0$
  \item $t_{34} = 0$
  \end{enumerate}
\item
$
  T^{(4)} = ||t^{(4)}_{ij}||, t^{(4)}_{ij} = t^{(3)}_{ij} \lor (t^{(3)}_{i4} \& t^{(3)}_{4j}) =
  \begin{pmatrix}
  1 && 1 && 1 && 0 \\
  1 && 1 && 1 && 0 \\
  1 && 1 && 1 && 0 \\
  1 && 1 && 1 && 1
  \end{pmatrix}
$
  \begin{enumerate}[label=]
  \item $t_{14} = 0$
  \item $t_{24} = 0$
  \item $t_{34} = 0$
  \end{enumerate}
\end{enumerate}
Ответ:
$
T =
\begin{pmatrix}
  1 && 1 && 1 && 0 \\
  1 && 1 && 1 && 0 \\
  1 && 1 && 1 && 0 \\
  1 && 1 && 1 && 1
\end{pmatrix}
$

\subsection*{2.}
$\overline{S} = T \& T^{T} =
\begin{pmatrix}
  1 && 1 && 1 && 0 \\
  1 && 1 && 1 && 0 \\
  1 && 1 && 1 && 0 \\
  1 && 1 && 1 && 1
\end{pmatrix}
\&
\begin{pmatrix}
  1 && 1 && 1 && 1 \\
  1 && 1 && 1 && 1 \\
  1 && 1 && 1 && 1 \\
  0 && 0 && 0 && 1
\end{pmatrix}
=
\begin{pmatrix}
  1 && 1 && 1 && 0 \\
  1 && 1 && 1 && 0 \\
  1 && 1 && 1 && 0 \\
  0 && 0 && 0 && 1
\end{pmatrix}
$
Ответ:
$
\overline{S} =
\begin{pmatrix}
  1 && 1 && 1 && 0 \\
  1 && 1 && 1 && 0 \\
  1 && 1 && 1 && 0 \\
  0 && 0 && 0 && 1
\end{pmatrix}
$

\subsection*{3.}
Вершины в первой строке $\overline{S}$ соотвествуют первой компоненте сильной связности,
следовательно первая компонента сильной связности -- $\{v_1, v_2, v_3\}$ \Rightarrow
$
\overline{S_1}
=
\begin{pmatrix}
  0 && 0 && 0 && 0 \\
  0 && 0 && 0 && 0 \\
  0 && 0 && 0 && 0 \\
  0 && 0 && 0 && 1
\end{pmatrix}
$
, вторая компонента -- $\{v_4\}$

Ответ:
$
\{v_1, v_2, v_3\}, \{v_4\}
$

\subsection*{4.}
Матрица контуров вычисляется как:
$
\overline{S} \& A
=
\begin{pmatrix}
  0 && 1 && 0 && 0 \\
  1 && 0 && 1 && 0 \\
  0 && 1 && 0 && 0 \\
  0 && 0 && 0 && 0
\end{pmatrix}
$

Ответ:
$
\begin{pmatrix}
  0 && 1 && 0 && 0 \\
  1 && 0 && 1 && 0 \\
  0 && 1 && 0 && 0 \\
  0 && 0 && 0 && 0
\end{pmatrix}
$

\end{document}

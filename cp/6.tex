\documentclass{article}

\usepackage{amsmath}
\usepackage[russian]{babel}
\usepackage[margin=2.5cm]{geometry}
\usepackage{booktabs}

\usepackage{tikz}
\usetikzlibrary{graphs, babel, quotes, calc, arrows.meta}

\setcounter{MaxMatrixCols}{20}

\title{Курсовая работа по дискретной математике\\Шестая задача}
\author{Клименко В. М. -- М8О-103Б-22 -- 11 вариант}
\date{Май, 2023}

\begin{document}
\maketitle

\section*{Дано}
\begin{center}
    \begin{tikzpicture}[circ/.style={circle, draw}]
        \path[nodes={circ}]
            (0, 2) node(1) {$v_1$}
            (1, 4) node(2) {$v_2$}
            (1, 0) node(3) {$v_3$}
            (4, 4) node(4) {$v_4$}
            (4, 0) node(5) {$v_5$}
            (5, 2) node(6) {$v_6$}
        ;
        \draw[nodes={circle, fill=white}, thick]
            (1) -- node{1} (2)
            (1) -- node{3} (3)
            (2) -- node{7} (3)
            (2) -- node{2} (4)
            (2) -- node[left=11pt, above=10pt]{8} (5)
            (3) -- node[left=11pt, below=10pt]{9} (4)
            (3) -- node{4} (5)
            (4) -- node{6} (6)
            (5) -- node{5} (6)
        ;
    \end{tikzpicture}
\end{center}


\section*{Задание}
Пусть каждому ребру неориентированного графа соответствует некоторый элемент 
электрической цепи. Составить линейно независимые системы уравнений Кирхгофа для
токов и напряжений. Пусть первому и пятому ребру соответствуют источники тока с ЭДС 
E1 и E2 (полярность выбирается произвольно), а остальные элементы являются сопротивлениями.
Используя закон Ома, и, предполагая внутренние сопротивления источников тока
равными нулю, получить систему уравнений для токов


\section*{Решение}
Находим остовное дерево $D =$
\begin{center}
    \begin{tikzpicture}[circ/.style={circle, draw}]
        \path[nodes={circ}]
            (0, 2) node(1) {$v_1$}
            (1, 4) node(2) {$v_2$}
            (1, 0) node(3) {$v_3$}
            (4, 4) node(4) {$v_4$}
            (4, 0) node(5) {$v_5$}
            (5, 2) node(6) {$v_6$}
        ;
        \draw[nodes={circle, fill=white}, thick]
            (1) -- node{1} (2)
            (1) -- node{3} (3)
            (2) -- node{2} (4)
            (3) -- node{4} (5)
            (5) -- node{5} (6)
        ;
    \end{tikzpicture}
\end{center}

Пусть ребру под номером $i$ будет соответствовать букве $q_i$, где $i \in 1..9$, тогда:
\begin{enumerate}
    \item $(D + q_6) = \mu_1 = v_1-v_2-v_4-v_6-v_5-v_3-v_1$
    \item $(D + q_7) = \mu_2 = v_1-v_2-v_3-v_1$
    \item $(D + q_8) = \mu_3 = v_1-v_2-v_5-v_3-v_1$
    \item $(D + q_9) = \mu_4 = v_1-v_2-v_4-v_3-v_1$
\end{enumerate}

Зададим произвольную ориентацию на графе:
\begin{center}
    \begin{tikzpicture}[circ/.style={circle, draw}]
        \path[nodes={circ}]
            (0, 2) node(1) {$v_1$}
            (1, 4) node(2) {$v_2$}
            (1, 0) node(3) {$v_3$}
            (4, 4) node(4) {$v_4$}
            (4, 0) node(5) {$v_5$}
            (5, 2) node(6) {$v_6$}
        ;
        \draw[-{Stealth[length=7pt]}] (1) -- (2);
        \draw[-{Stealth[length=7pt]}] (1) -- (3);
        \draw[-{Stealth[length=7pt]}] (2) -- (3);
        \draw[-{Stealth[length=7pt]}] (2) -- (4);
        \draw[-{Stealth[length=7pt]}] (2) -- (5);
        \draw[-{Stealth[length=7pt]}] (3) -- (4);
        \draw[-{Stealth[length=7pt]}] (3) -- (5);
        \draw[-{Stealth[length=7pt]}] (4) -- (6);
        \draw[-{Stealth[length=7pt]}] (5) -- (6);
        \draw[nodes={circle, fill=white}]
            (1) -- node{1} (2)
            (1) -- node{3} (3)
            (2) -- node{7} (3)
            (2) -- node{2} (4)
            (2) -- node[left=11pt, above=10pt]{8} (5)
            (3) -- node[left=11pt, below=10pt]{9} (4)
            (3) -- node{4} (5)
            (4) -- node{6} (6)
            (5) -- node{5} (6)
        ;
    \end{tikzpicture}
\end{center}

Получаем цикломатическую матрицу
$$
C =
\begin{pmatrix}
    1 && 1 && -1 && -1 && -1 && 1 && 0 && 0 &&  0 \\
    1 && 0 && -1 &&  0 &&  0 && 0 && 1 && 0 &&  0 \\
    1 && 0 && -1 && -1 &&  0 && 0 && 0 && 1 &&  0 \\
    1 && 1 && -1 &&  0 &&  0 && 0 && 0 && 0 && -1
\end{pmatrix}
$$

По закону Кирхгофа для напряжения:
\begin{align*}
    C \times U = 0 = 
    \begin{pmatrix}
        1 && 1 && -1 && -1 && -1 && 1 && 0 && 0 &&  0 \\
        1 && 0 && -1 &&  0 &&  0 && 0 && 1 && 0 &&  0 \\
        1 && 0 && -1 && -1 &&  0 && 0 && 0 && 1 &&  0 \\
        1 && 1 && -1 &&  0 &&  0 && 0 && 0 && 0 && -1
    \end{pmatrix}
    \times
    \begin{pmatrix}
        u_1 \\ \dots \\ u_9
    \end{pmatrix}
    \Longrightarrow \\
    \Longrightarrow
    \begin{cases}
        u_1 + u_2 - u_3 - u_4 - u_5 + u_6 = 0 \\
        u_1 - u_3 + u_7 = 0 \\
        u_1 - u_3 - u_4 + u_8 = 0 \\
        u_1 + u_2 - u_9 = 0
    \end{cases}
    \Longrightarrow
    \begin{cases}
        u_6 = -u_1 - u_2 + u_3 + u_4 + u_5 \\
        u_7 = -u_1 + u_3 \\
        u_8 = -u_1 + u_3 + u_4 \\
        u_9 = u_1 + u_2
    \end{cases}
    \Longrightarrow
    u_6, u_7, u_8, u_9 - \text{базис} \\
\end{align*}

Составим матрицу инцидентности
\begin{equation*}
    \begin{split}
        B =
        \begin{pmatrix}
            -1 &&  0 && -1 &&  0 &&  0 &&  0 &&  0 &&  0 &&  0 \\
             1 && -1 &&  0 &&  0 &&  0 &&  0 && -1 && -1 &&  0 \\
             0 &&  0 &&  1 && -1 &&  0 &&  0 &&  1 &&  0 && -1 \\
             0 &&  1 &&  0 &&  0 &&  0 && -1 &&  0 &&  0 &&  1 \\
             0 &&  0 &&  0 &&  1 && -1 &&  0 &&  0 &&  1 &&  0 \\
             0 &&  0 &&  0 &&  0 &&  1 &&  1 &&  0 &&  0 &&  0
        \end{pmatrix}
        \sim \\ \sim
        \begin{pmatrix}
            1 && 0 && 0 && 0 && 0 && -1 && -1 && -1 && 0 \\
            0 && 1 && 0 && 0 && 0 && -1 &&  0 &&  0 && 0 \\
            0 && 0 && 1 && 0 && 0 &&  1 &&  1 &&  1 && 0 \\
            0 && 0 && 0 && 1 && 0 &&  1 &&  0 &&  1 && 0 \\
            0 && 0 && 0 && 0 && 1 &&  1 &&  0 &&  0 && 0 \\
            0 && 0 && 0 && 0 && 0 &&  0 &&  0 &&  0 && 1
        \end{pmatrix}
    \end{split}
\end{equation*}

По закону Кирхгофа для тока:
\begin{align*}
    B \times I = 0 = 
    \begin{pmatrix}
        1 && 0 && 0 && 0 && 0 && -1 && -1 && -1 && 0 \\
        0 && 1 && 0 && 0 && 0 && -1 &&  0 &&  0 && 0 \\
        0 && 0 && 1 && 0 && 0 &&  1 &&  1 &&  1 && 0 \\
        0 && 0 && 0 && 1 && 0 &&  1 &&  0 &&  1 && 0 \\
        0 && 0 && 0 && 0 && 1 &&  1 &&  0 &&  0 && 0 \\
        0 && 0 && 0 && 0 && 0 &&  0 &&  0 &&  0 && 1
    \end{pmatrix}
    \times
    \begin{pmatrix}
        i_1 \\ \dots \\ i_9
    \end{pmatrix}
    \Longrightarrow \\ \Longrightarrow
    \begin{cases}
        i_1 = i_6 + i_7 + i_8 \\
        i_2 = i_6 \\
        i_3 = -i_6 - i_7 - i_8 \\
        i_4 = -i_6 - i_8 \\
        i_5 = -i_6 \\
        i_9 = 0
    \end{cases}
    \Longrightarrow
    \begin{cases}
        i_1 = i_6 + i_7 + i_8 \\
        i_2 = i_6 \\
        i_3 = -i_6 - i_7 - i_8 \\
        i_4 = -i_6 - i_8 \\
        i_5 = -i_6 \\
        i_9 = 0 \\
        u_6 = -u_1 - u_2 + u_3 + u_4 + u_5 \\
        u_7 = -u_1 + u_3 \\
        u_8 = -u_1 + u_3 + u_4 \\
        u_9 = u_1 + u_2
    \end{cases}
    \Longrightarrow \\ \Longrightarrow
    \begin{cases}
        i_1 = i_6 + i_7 + i_8 \\
        i_2 = i_6 \\
        i_3 = -i_6 - i_7 - i_8 \\
        i_4 = -i_6 - i_8 \\
        i_5 = -i_6 \\
        i_9 = 0 \\
        i_6r_6 = -e_1 - i_2r_2 + i_3r_3 + i_4r_4 + e_5 \\
        i_7r_7 = -e_1 + i_3r_3 \\
        i_8r_8 = -e_1 + i_3r_3 + i_4r_4 \\
        i_9r_9 = e_1 + i_2r_2
    \end{cases}
    \Longrightarrow
    \begin{cases}
        i_1 = i_6 + i_7 + i_8 \\
        i_2 = i_6 \\
        i_3 = -i_6 - i_7 - i_8 \\
        i_4 = -i_6 - i_8 \\
        i_5 = -i_6 \\
        i_9 = 0 \\
        i_6r_6 = i_3r_3 + i_4r_4 + e_5 \\
        i_7r_7 = -e_1 + i_3r_3 \\
        i_8r_8 = -e_1 + i_3r_3 + i_4r_4 \\
        e_1 = -i_2r_2
    \end{cases}
\end{align*}


\section*{Ответ}
\begin{equation*}
    \begin{cases}
        i_1 = i_6 + i_7 + i_8 \\
        i_2 = i_6 \\
        i_3 = -i_6 - i_7 - i_8 \\
        i_4 = -i_6 - i_8 \\
        i_5 = -i_6 \\
        i_9 = 0 \\
        i_6r_6 = i_3r_3 + i_4r_4 + e_5 \\
        i_7r_7 = -e_1 + i_3r_3 \\
        i_8r_8 = -e_1 + i_3r_3 + i_4r_4 \\
        e_1 = -i_2r_2
    \end{cases}
\end{equation*}

\end{document}

\documentclass{article}

\usepackage{amsmath}
\usepackage[russian]{babel}
\usepackage[margin=2.5cm]{geometry}

\setcounter{MaxMatrixCols}{20}

\title{Курсовая работа по дискретной математике\\Вторая задача}
\author{Клименко В. М. -- М8О-103Б-22 -- 11 вариант}
\date{Март, 2023}

\begin{document}
\maketitle


\section*{Дано}
Матрица длин дуг $A$:
$$
\begin{pmatrix}
    \infty && 2 && \infty && 5 && \infty && 6 && \infty && \infty \\
    6 && \infty && 12 && 3 && \infty && \infty && \infty && \infty \\
    7 && \infty && \infty && \infty && 1 && \infty && \infty && 1 \\
    5 && 3 && \infty && \infty && 6 && 2 && \infty && \infty \\
    \infty && \infty && 1 && \infty && \infty && \infty && 3 && 4 \\
    3 && \infty && \infty && 2 && \infty && \infty && 2 && \infty \\
    \infty && \infty && \infty && \infty && 3 && \infty && \infty && 6 \\
    8 && \infty && \infty && \infty && 13 && \infty && \infty && \infty \\
\end{pmatrix}
$$


\section*{Задание}
Используя алгоритм Форда, найти минимальные пути из первой вершины во
все достижимые вершины в нагруженном графе, заданном матрицей длин дуг $A$.


\section*{Решение}



\section*{Ответ}


\end{document}

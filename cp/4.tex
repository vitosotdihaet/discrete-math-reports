\documentclass{article}

\usepackage{amsmath}
\usepackage[russian]{babel}
\usepackage[margin=2.5cm]{geometry}
\usepackage{booktabs}

\usepackage{tikz}
\usetikzlibrary{tikzmark}

\newcommand\tm[1]{\tikzmark{#1}}
\newcommand\ld[2]{$\lambda^{(#1)}_{#2}$}

\setcounter{MaxMatrixCols}{20}

\title{Курсовая работа по дискретной математике\\Четвертая задача}
\author{Клименко В. М. -- М8О-103Б-22 -- 11 вариант}
\date{Март, 2023}

\begin{document}
\maketitle


\section*{Дано}
Матрица длин дуг $A$:
$$
\begin{pmatrix}
    \infty && 2      && \infty  && 5      && \infty && 6      && \infty && \infty \\
    6      && \infty && 12      && 3      && \infty && \infty && \infty && \infty \\
    7      && \infty && \infty  && \infty && 1      && \infty && \infty && 1      \\
    5      && 3      && \infty  && \infty && 6      && 2      && \infty && \infty \\
    \infty && \infty && 1       && \infty && \infty && \infty && 3      && 4      \\
    3      && \infty && \infty  && 2      && \infty && \infty && 2      && \infty \\
    \infty && \infty && \infty  && \infty && 3      && \infty && \infty && 6      \\
    8      && \infty && \infty  && \infty && 13     && \infty && \infty && \infty \\
\end{pmatrix}
$$


\section*{Задание}
Используя алгоритм Форда, найти минимальные пути из первой вершины во
все достижимые вершины в нагруженном графе, заданном матрицей длин дуг $A$.


\section*{Решение}
\begin{tabular}{l|llllllll|llllllll}
             & $V1$     & $V2$     & $V3$     & $V4$     & $V5$     & $V6$     & $V7$     & $V8$     
    & \ld{0}{i}& \ld{1}{i}& \ld{2}{i}& \ld{3}{i}& \ld{4}{i}& \ld{5}{i}& \ld{6}{i}& \ld{7}{i}\\
    \midrule
    $V1$     & $\infty$ & 2        & $\infty$ & 5        & $\infty$ & 6        & $\infty$ & $\infty$
             & 0\tm{a}  & 0        & 0        & 0        & 0        & 0        & 0        & 0        \\
    $V2$     & 6        & $\infty$ & 12       & 3        & $\infty$ & $\infty$ & $\infty$ & $\infty$
             & $\infty$ & 2\tm{b}  & 2        & 2        & 2        & 2        & 2        & 2        \\
    $V3$     & 7        & $\infty$ & $\infty$ & $\infty$ & 1        & $\infty$ & $\infty$ & 1        
             & $\infty$ & $\infty$ & 14       & 12\tm{g} & 12       & 12       & 12       & 12       \\
    $V4$     & 5        & 3        & $\infty$ & $\infty$ & 6        & 2        & $\infty$ & $\infty$ 
             & $\infty$ & 5\tm{c}  & 5        & 5        & 5        & 5        & 5        & 5        \\
    $V5$     & $\infty$ & $\infty$ & 1        & $\infty$ & $\infty$ & $\infty$ & 3        & 4        
             & $\infty$ & $\infty$ & 11\tm{f} & 11       & 11       & 11       & 11       & 11       \\
    $V6$     & 3        & $\infty$ & $\infty$ & 2        & $\infty$ & $\infty$ & 2        & $\infty$ 
             & $\infty$ & 6\tm{d}  & 6        & 6        & 6        & 6        & 6        & 6        \\
    $V7$     & $\infty$ & $\infty$ & $\infty$ & $\infty$ & 3        & $\infty$ & $\infty$ & 6        
             & $\infty$ & $\infty$ & 8\tm{e}  & 8        & 8        & 8        & 8        & 8        \\
    $V8$     & 8        & $\infty$ & $\infty$ & $\infty$ & 13       & $\infty$ & $\infty$ & $\infty$ 
             & $\infty$ & $\infty$ & $\infty$ & 14\tm{h} & 14       & 14       & 14       & 14       \\
\end{tabular}

\begin{tikzpicture}[overlay, remember picture, transform canvas={yshift=.25\baselineskip}]
    \draw[->] ([xshift=8pt]pic cs:a) -- ([xshift=-8pt]pic cs:b);
    \draw[->] ([xshift=8pt]pic cs:a) -- ([xshift=-8pt]pic cs:c);
    \draw[->] ([xshift=8pt]pic cs:d) -- ([xshift=-8pt]pic cs:e);
    \draw[->] ([xshift=8pt]pic cs:c) -- ([xshift=-13pt]pic cs:f);
    \draw[->] ([xshift=3pt]pic cs:f) -- ([xshift=-13pt]pic cs:g);
    \draw[->] ([xshift=3pt]pic cs:e) -- ([xshift=-13pt]pic cs:h);
\end{tikzpicture}
\\
\begin{enumerate}
    \item Из $v_1$ в $v_2$ - $v_1 - v_2$, длина равна 2
    \begin{enumerate}
        \item \ld{0}{1}$+ c_{12} = 0 + 2 =$\ld{1}{2}
    \end{enumerate}
    
    \item Из $v_1$ в $v_3$ - $v_1 - v_4 - v_5 - v_3$, длина равна 12
    \begin{enumerate}
        \item \ld{0}{1}$+ c_{14} = 0 + 5 =$\ld{1}{4}
        \item \ld{1}{4}$+ c_{45} = 5 + 6 =$\ld{2}{5}
        \item \ld{2}{5}$+ c_{53} = 11 + 1 =$\ld{3}{3}
    \end{enumerate}
    
    \item Из $v_1$ в $v_4$ - $v_1 - v_4$, длина равна 5
    \begin{enumerate}
        \item \ld{0}{1}$+ c_{14} = 0 + 5 =$\ld{1}{4}
    \end{enumerate}

    \item Из $v_1$ в $v_5$ - $v_1 - v_4 - v_5$, длина равна 11
    \begin{enumerate}
        \item \ld{0}{1}$+ c_{14} = 0 + 5 =$\ld{1}{4}
        \item \ld{1}{4}$+ c_{45} = 5 + 6 =$\ld{2}{5}
    \end{enumerate}

    \item Из $v_1$ в $v_6$ - $v_1 - v_6$, длина равна 6
    \begin{enumerate}
        \item \ld{0}{1}$+ c_{16} = 0 + 6 =$\ld{1}{6}
    \end{enumerate}

    \item Из $v_1$ в $v_7$ - $v_1 - v_6 - v_7$, длина равна 8
    \begin{enumerate}
        \item \ld{0}{1}$+ c_{16} = 0 + 6 =$\ld{1}{6}
        \item \ld{1}{6}$+ c_{67} = 6 + 2 =$\ld{2}{7}
    \end{enumerate}

    \item Из $v_1$ в $v_8$ - $v_1 - v_6 - v_7 - v_8$, длина равна 14
    \begin{enumerate}
        \item \ld{0}{1}$+ c_{16} = 0 + 6 =$\ld{1}{6}
        \item \ld{1}{6}$+ c_{67} = 6 + 2 =$\ld{2}{7}
        \item \ld{2}{7}$+ c_{78} = 8 + 6 =$\ld{3}{8}
    \end{enumerate}
\end{enumerate}


\end{document}

\documentclass{article}

\usepackage{amsmath}
\usepackage[russian]{babel}
\usepackage[margin=2.5cm]{geometry}
\usepackage{booktabs}

\usepackage{tikz}
\usetikzlibrary{tikzmark}

\newcommand\tm[1]{\tikzmark{#1}}
\newcommand\ld[2]{$\lambda^{(#1)}_{#2}$}

\setcounter{MaxMatrixCols}{20}

\title{Курсовая работа по дискретной математике\\Четвертая задача}
\author{Клименко В. М. -- М8О-103Б-22 -- 11 вариант}
\date{Март, 2023}

\begin{document}
\maketitle


\section*{Дано}
Матрица длин дуг $A$:
$$
\begin{pmatrix}
    \infty && 2      && \infty  && 5      && \infty && 6      && \infty && \infty \\
    6      && \infty && 12      && 3      && \infty && \infty && \infty && \infty \\
    7      && \infty && \infty  && \infty && 1      && \infty && \infty && 1      \\
    5      && 3      && \infty  && \infty && 6      && 2      && \infty && \infty \\
    \infty && \infty && 1       && \infty && \infty && \infty && 3      && 4      \\
    3      && \infty && \infty  && 2      && \infty && \infty && 2      && \infty \\
    \infty && \infty && \infty  && \infty && 3      && \infty && \infty && 6      \\
    8      && \infty && \infty  && \infty && 13     && \infty && \infty && \infty \\
\end{pmatrix}
$$


\section*{Задание}
Используя алгоритм Форда, найти минимальные пути из первой вершины во
все достижимые вершины в нагруженном графе, заданном матрицей длин дуг $A$.


\section*{Решение}
\subsubsection*{Пункт 1}
\begin{tabular}{l|llllllll|llllllll}
             & $V1$     & $V2$     & $V3$     & $V4$     & $V5$     & $V6$     & $V7$     & $V8$     
    & \ld{0}{i}& \ld{1}{i}& \ld{2}{i}& \ld{3}{i}& \ld{4}{i}& \ld{5}{i}& \ld{6}{i}& \ld{7}{i}\\
    \midrule
    $V1$     & $\infty$ & 2        & $\infty$ & 5        & $\infty$ & 6        & $\infty$ & $\infty$
             & 0\tm{a}  & 0        & 0        & 0        & 0        & 0        & 0        & 0        \\
    $V2$     & 6        & $\infty$ & 12       & 3        & $\infty$ & $\infty$ & $\infty$ & $\infty$
             & $\infty$ & 2\tm{b}  & \\
    $V3$     & 7        & $\infty$ & $\infty$ & $\infty$ & 1        & $\infty$ & $\infty$ & 1        
             & $\infty$ & $\infty$ & \\
    $V4$     & 5        & 3        & $\infty$ & $\infty$ & 6        & 2        & $\infty$ & $\infty$ 
             & $\infty$ & 5\tm{c}  & \\
    $V5$     & $\infty$ & $\infty$ & 1        & $\infty$ & $\infty$ & $\infty$ & 3        & 4        
             & $\infty$ & $\infty$ & \\
    $V6$     & 3        & $\infty$ & $\infty$ & 2        & $\infty$ & $\infty$ & 2        & $\infty$ 
             & $\infty$ & 6        & \\
    $V7$     & $\infty$ & $\infty$ & $\infty$ & $\infty$ & 3        & $\infty$ & $\infty$ & 6        
             & $\infty$ & $\infty$ & \\
    $V8$     & 8        & $\infty$ & $\infty$ & $\infty$ & 13       & $\infty$ & $\infty$ & $\infty$ 
             & $\infty$ & $\infty$ & \\
\end{tabular}

\begin{tikzpicture}[overlay, remember picture, shorten >=.5pt, shorten <=.5pt, transform canvas={yshift=.25\baselineskip}]
    \draw[->] ([xshift=8pt]pic cs:a) -- ([xshift=-8pt]pic cs:b);
    \draw[->] ([xshift=8pt]pic cs:a) -- ([xshift=-8pt]pic cs:c);
\end{tikzpicture}



\section*{Ответ}


\end{document}

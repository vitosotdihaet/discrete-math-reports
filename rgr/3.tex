\documentclass{article}

\usepackage{amsmath}
\usepackage[russian]{babel}
\usepackage[margin=2.5cm]{geometry}
\usepackage{booktabs}

\newcommand\headercell[1]{\smash[b]{\begin{tabular}[t]{@{}c@{}} #1 \end{tabular}}}


\title{РГР по дискретной математике\\Третья задача}
\author{Клименко В. М. -- М8О-103Б-22 -- 11 вариант}
\date{Март, 2023}

\begin{document}
\maketitle


\section*{Дано}
$H = <(12), (34)>$


\section*{Задание}
Определить для заданной подгруппы $H \subset S_4$:
\begin{enumerate}
    \item элементы из $H$
    \item левые смежные классы группы $S_4$ по $H$
    \item правые смежные классы группы $S_4$ по $H$
    \item является ли $H$ нормальной подгруппой
\end{enumerate}


\section*{Решение}
% $S_4 = \{
% \pi_0,
% (12), (13), (14), (23), (24), (34),
% (123), (124), (132), (134), (142), (143), (234), (243),\\
% (1234), (1243), (1324), (1342), (1423), (1432)
% \}$

\subsection*{Пункт 1}
\begin{tabular}{l|lll|l}
    $\cdot$      & $pi_0$     & $(12)$     & $(34)$     & $(12)(34)$ \\ 
    \midrule
      $pi_0$     & $pi_0$     & $(12)$     & $(34)$     & $(12)(34)$ \\
      $(12)$     & $(12)$     & $pi_0$     & $(12)(34)$ & $(34)$     \\
      $(34)$     & $(34)$     & $(34)(12)$ & $pi_0$     & $pi_0$     \\
    \midrule
      $(12)(34)$ & $(12)(34)$ & $pi_0$     & $pi_0$     & $pi_0$     \\
\end{tabular}
\\\\
$H = \{\pi_0, (12), (34), (12)(34)\}$

\subsection*{Пункт 2}
\begin{enumerate}
    \item $\pi_0 \cdot H = \{\pi_0, (12) , (34) , (12)(34)\}$
    \item $(13)  \cdot H = \{(13) , (123), (341), (3412)  \}$
    \item $(14)  \cdot H = \{(14) , (124), (314), (3124)  \}$
    \item $(23)  \cdot H = \{(23) , (132), (342), (3421)  \}$
    \item $(24)  \cdot H = \{(24) , (142), (32) , (3214)  \}$
    \item $(123) \cdot H = \{(123), (13) , (3412), (341)  \}$
\end{enumerate}

\subsection*{Пункт 3}
\begin{enumerate}
    \item $H \cdot \pi_0 = \{\pi_0, (12) , (34) , (12)(34)\}$
    \item $H \cdot (13)  = \{(13) , (132), (143), (1432)  \}$
    \item $H \cdot (14)  = \{(14) , (142), (134), (1342)  \}$
    \item $H \cdot (23)  = \{(23) , (231), (243), (2431)  \}$
    \item $H \cdot (24)  = \{(24) , (241), (234), (2341)  \}$
    \item $H \cdot (123) = \{(123), \pi_0, (1243), \pi_0  \}$
\end{enumerate}

\subsection*{Пункт 4}
$H$ не является нормальной подгруппой, так как ЛСК $\ne$ ПСК


\end{document}

\documentclass{article}

\usepackage{amsmath}
\usepackage[russian]{babel}
\usepackage[margin=2.5cm]{geometry}


\title{Курсовая работа по дискретной математике\\Третья задача}
\author{Клименко В. М. -- М8О-103Б-22 -- 11 вариант}
\date{Март, 2023}

\begin{document}
\maketitle


\section*{Дано}
$H = <(12), (34)>$


\section*{Задание}
Определить для заданной подгруппы $H \subset S_4$:
\begin{enumerate}
    \item элементы из $H$
    \item левые смежные классы группы $S_4$ по $H$
    \item правые смежные классы группы $S_4$ по $H$
    \item является ли $H$ нормальной подгруппой
\end{enumerate}


\section*{Решение}
\subsection*{Пункт 1}
$H = \{\pi_0, (12),(34)\}$

\subsection*{Пункт 2}
\begin{enumerate}
    \item $\pi_0 \cdot H = \{\pi_0, (12) , (34)\}$
    \item $(13)  \cdot H = \{(13) , (123), (341)\}$
    \item $(14)  \cdot H = \{(14) , (124), (314)\}$
    \item $(23)  \cdot H = \{(23) , (132), (342)\}$
\end{enumerate}

\subsection*{Пункт 3}
\begin{enumerate}
    \item $H \cdot \pi_0 = \{\pi_0, (12) , (34)\}$
    \item $H \cdot (13)  = \{(13) , (132), (143)\}$
    \item $H \cdot (14)  = \{(14) , (142), (134)\}$
    \item $H \cdot (23)  = \{(23) , (231), (243)\}$
\end{enumerate}

\subsection*{Пункт 4}
$H$ не является нормальной подгруппой, так как ЛСК $\ne$ ПСК


\end{document}

\documentclass{article}

\usepackage{amsmath}
\usepackage[russian]{babel}
\usepackage[margin=2.5cm]{geometry}
\usepackage{amssymb}


\title{РГР по дискретной математике\\Пятая задача}
\author{Клименко В. М. -- М8О-103Б-22 -- 11 вариант}
\date{Апрель, 2023}

\begin{document}
\maketitle


\section*{Дано}
Квадратные матрицы порядка 4: <M, +, $\times$> с элементами из $\mathbb{R}$


\section*{Задание}
Определить, является ли полем или кольцом заданная алгебраическая структура.
Проверить, существуют ли делители нуля.

\section*{Решение}
\subsection*{Сложение}
\begin{enumerate}
    \item коммутативность, ассоциативность, замкнутость - очевидно по свойствам матриц
    \item единичный элемент -- $(0)$ =
    $
    \begin{pmatrix}
        0 && 0 && 0 && 0 \\
        0 && 0 && 0 && 0 \\
        0 && 0 && 0 && 0 \\
        0 && 0 && 0 && 0
    \end{pmatrix}
    $
    \item обратный элемент -- $A^{-1}_+ = -A = -1 \times A$: $A + (-A) = (0)$
\end{enumerate}

\subsection*{Умножение}
\begin{enumerate}
    \item коммутативность -- не выполняется
    \item ассоциативность, замкнутость -- очевидно по свойствам матриц
    \item единичный элемент -- $E$ =
    $
    \begin{pmatrix}
        1 && 0 && 0 && 0 \\
        0 && 1 && 0 && 0 \\
        0 && 0 && 1 && 0 \\
        0 && 0 && 0 && 1
    \end{pmatrix}
    $
    \item обратный элемент -- 
    $A^{-1}_{\times} = A^{-1} = \frac{1}{\Delta A} \times (A_{ij})^T$:
    $A \times A^{-1} = E$ существует только если детерминант матрицы не равен нулю
\end{enumerate}

\subsection*{Дистрибутивность}
$A\times(B+C) = A\times B+ A\times C$ -- по дистрибутивности матриц
% здесь надо сказать еще дистрибутивность (A + B)С = AC + BC

\subsection*{Делители нуля}
делители нуля существуют, например:
$$
\begin{pmatrix}
    1 && 0 && 0 && 0 \\
    0 && 0 && 0 && 0 \\
    0 && 0 && 0 && 0 \\
    0 && 0 && 0 && 0
\end{pmatrix}
\times
\begin{pmatrix}
    0 && 0 && 0 && 0 \\
    0 && 0 && 0 && 0 \\
    0 && 0 && 0 && 0 \\
    0 && 0 && 0 && 1 \\
\end{pmatrix}
=
\begin{pmatrix}
    0 && 0 && 0 && 0 \\
    0 && 0 && 0 && 0 \\
    0 && 0 && 0 && 0 \\
    0 && 0 && 0 && 0
\end{pmatrix}
$$

\section*{Ответ}
Алгебраическая структура квадратные матрицы порядка 4: <M, +, $\times$>
с элементами из $\mathbb{R}$ является \textbf{кольцом}


\end{document}

\documentclass{article}

\usepackage{amsmath}
\usepackage{enumitem}

\usepackage[T1]{fontenc}
\usepackage[russian]{babel}

\title{РГР по дискретной математики\\Первая задача}
\author{Клименко В. М. -- М8О-103Б-22 -- 11 вариант}
\date{Март, 2023}

\begin{document}

\maketitle

\section*{Дано}
Оператор примитивной рекурсии, оператор суперпозиции, а также функции

$$
\begin{align}
  &S(x) = x + 1\\
  &O(x) = 0\\
  &I^{n}_{m}(x_1, . . . x_n) = x_m$, где $1 \le m \le n\\
  &\sigma(x_1, x_2) = x_1 + x_2
\end{align}
$$

\section*{Задание}
Получить функцию $f(x,y) = (2x+3)y$

\section*{Решение}
$$
\begin{cases}
    f(x, 0) = O(x)\\
    f(x, y+1) = (2x+3)(y+1) = 2xy+2x+3y+3 = z+2x = \sigma(z,S(x,x)),
\end{cases}
$где $ z=2xy+3y+3
$$

\end{document}
